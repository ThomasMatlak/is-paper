%!TEX root = ../username.tex
\chapter{A Markov Model for Melody Generation} \label{melody:markov}

\section{Setup of the Model}

In this model, we use two Markov processes: one to determine interval changes between notes to generate pitches, and the other to generate rhythms.
To create the interval transition matrix, the source is iterated, keeping track of the previous interval at each step, incrementing the matrix at position $a_{i,j}$ for the previous interval $i$ and the current interval $j$.
The rhythmic transition matrix is similarly generated.

Because the number of possible intervals and rhythms are theoretically infinite, we do not want to try to store a matrix capable of holding every possible transition.
Rather, in this implementation, we use a Python dictionary.
More specifically, we use nested \lstinline[columns=fixed]{defaultdict}s to store the matrices.

An example rhythmic transition matrix is given in Figure \ref{fig:rhythmTransitionMatrix}, based on the \textit{Little Fugue in g minor} by J.S. Bach.

\begin{figure}
	\centering
	\begin{tabular}{c | c c c c c c c c}
		& 0.25 & 0.5 & 0.75 & 1.0 & 1.25 & 1.5 & 2.0 & 2.25\\
		\hline
		0.25 & 606 & 21 & 1 & 2 & 1 & 2 & 0 & 1\\
		0.5 & 23 & 109 & 0 & 7 & 1 & 0 & 0 & 0\\
		0.75 & 1 & 0 & 0 & 0 & 0 & 0 & 0 & 0\\
		1.0 & 1 & 6 & 0 & 3 & 0 & 3 & 2 & 0\\
		1.25 & 2 & 0 & 0 & 0 & 0 & 0 & 0 & 0\\
		1.5 & 0 & 5 & 0 & 0 & 0 & 0 & 0 & 0\\
		2.0 & 0 & 0 & 0 & 2 & 0 & 0 & 0 & 0\\
		2.25 & 1 & 0 & 0 & 0 & 0 & 0 & 0 & 0
	\end{tabular}
	\caption{An example rhythmic transition matrix. A value of $1.0$ indicates a quarter note.}
	\label{fig:rhythmTransitionMatrix}
\end{figure}

\section{Generation}

After the transition matrices are created, the program randomly selects a rhythmic value, based on the probabilities that correspond with the previous note, and a pitch.
The pitch value comes from the interval from the previous note.
The interval is chosen based on the probabilities that correspond with the previous interval.

The program stops generating notes when the stopping criteria are met: the generated melody is at least eight measures, the melody ends at the end of a measure, and the last note is a tonic chord tone.
Because the generated melody could theoretically be infinite, the program terminates generation after one-hundred measures are generated without meeting the stopping criteria.
