%!TEX root = ../username.tex
\chapter{The Software} \label{software}

% TODO give more details

This chapter discusses the various components of the software used in this project.

All components use \texttt{music21}, a Python library for music developed by MIT, to handle input and output of music notation to and from a Python-native format.

\section{Markov Melody Generator}

In the code repository for this project, the directory \texttt{intervalMarkovChain} contains Python code to generate melodies using Markov chains of arbitrary order.
The theory behind this software component is discussed in section \ref{bg:markov} and Chapter \ref{markov}.
The software uses two Markov chains, one for the intervals between notes, and the other for rhythms.

The file \texttt{intervalMarkovChain/markovChain.py} contains a class to create these Markov chains.
An $n$th order Markov chain is represented using \texttt{defaultdict}s nested $n + 1$ deep.

The file \texttt{intervalMarkovChain/intervalMarkovChain.py} contains code to read a corpus of music, generate the Markov chains using the \texttt{MarkovChain} class, and create a new melody with them.

\section{Genetic Algorithms}

The directory \texttt{geneticAlgorithms} contains Python code to manipulate existing melodies with genetic algorithms to produce better output.
The theory behind this software component is in section \ref{bg:ga} and Chapter \ref{ga}.

\subsection{Fitness Function}

This program uses a neural network created using TensorFlow as a surrogate fitness.
The file \texttt{geneticAlgorithms/fitnessFunctionTraining.py} contains code to train the neural network and save it for later use, as well as example code for how to run a specific melody segment through the neural network to determine its fitness.

\subsection{Mutations}

The file \texttt{geneticAlgorithms/genetic.py} contains functions to perform mutation on \texttt{music21} streams.
Section \ref{ga:mutate} contains the theoretical background of these mutations.

\subsection{Evolving Melodies}



\section{Harmonization and Counterpoint}
