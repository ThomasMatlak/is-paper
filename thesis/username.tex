%%%%%%%%%%%%%%%%%%%%%%%%%%%%%%%%%%%%%%%%%%%%%%%%%%%%%%%
% use this declaration for a draft  version of your IS
\documentclass[10pt,palatino,code,picins,kaukecopyright,openright,woolshort,dropcaps,verbatim,index,euler]{woosterthesis}
%\documentclass[10pt,code,picins,kaukecopyright,openright,woolshort,dropcaps,verbatim,euler,index,colophon,blacklinks,twoside]{woosterthesis}
% note that you can specify the woosterchicago option to use Chicago citation style and achemso to use the American Chemical Society citation format
%
%%%%%%%%%%%%%%%%%%%%%%%%%%%%%%%%%%%%%%%%%%%%%%%%%%%%%%%
%
% use this declaration for the print version of your IS
%\documentclass[12pt,code,palatino,picins,blacklinks,kaukecopyright,openright,twoside]{woosterthesis} % probably what most students would use
%
%%%%%%%%%%%%%%%%%%%%%%%%%%%%%%%%%%%%%%%%%%%%%%%%%%%%%%%
%
% use this declaration for the PDF version of your IS
%\documentclass[12pt,code,palatino,picins,kaukecopyright,openright,twoside]{woosterthesis}
%
%%%%%%%%%%%%%%%%%%%%%%%%%%%%%%%%%%%%%%%%%%%%%%%%%%%%%%%

\usepackage{wasysym}

\input{styles/packages}

\input{styles/personal}

\input{styles/theorems}

\setcounter{secnumdepth}{5}% controls the numbering of sections
\setcounter{tocdepth}{6}% controls the number of levels in the Contents

\title{Algorithmic Music Composition}
\thesistype{Independent Study Thesis}
\author{Thomas Matlak}
\degreetoobtain{B.A.}
\presentschool{The College of Wooster}
\academicprogram{Departments of Computer Science and Mathematics}
\gradyear{2018}
\advisor{Nathan Sommer (Computer Science)}
\secondadvisor{Nathan Fox (Mathematics)}
%\reader{Reader}
\copyrighted   
%\copyrightdate{}                  
\makeindex % comment this line if you do not have an index

\begin{document}

%%%%%%%%%%%%%%%%%%%%%%%%%%%%%%%%%%%%%%%%%%%%%%%%%%%%%%%
%
%  The front matter includes acknowledgments, dedications, vitas, list of tables, list of figures,
%  copyright, abstract, title page, and contents.
%
%%%%%%%%%%%%%%%%%%%%%%%%%%%%%%%%%%%%%%%%%%%%%%%%%%%%%%%

\frontmatter
\maketitle
\ClearShipoutPicture
\clearpage\thispagestyle{empty}\null\clearpage
\disscopyright 

% Abstract

\begin{abstract} \label{abstract}
This is a short summary of my thesis, including very basic background information, a simplified description of my work, and an overview of the results.
\end{abstract}

% Dedications

\dedication{This work is dedicated to the future generations of Wooster students.}

% Acknowledgments

\begin{acknowl}  

\end{acknowl}

% Vita

\begin{vita} 
% You talk about yourself and how you got to where you are now. There is a structured form for the Vita that can be used if you want, but I don't encourage it.

%%%%%%%%%%%%%%%%%%%%%%%%%%%%%%%%%%%%%%%%%%%%%%%%%%%%%%%
%
%  The list below is for a thesis that requires a more structured Vita such as a masters or Ph.D.
%
%%%%%%%%%%%%%%%%%%%%%%%%%%%%%%%%%%%%%%%%%%%%%%%%%%%%%%%

%\begin{datelist}
%\item[April 6, 1970]Born-Wooster, Ohio
%\item[August 11, 1990]Chosen to present an undergraduate paper at the 75th meeting of the MAA, Columbus, Ohio
%\item[August 1990--August 1991]President Wooster Student Chapter of the MAA, The College of Wooster, Wooster, Ohio
%\item[August 1991--May 1992]Secretary Wooster Student Chapter of the MAA, The College of Wooster, Wooster, Ohio
%\item[1992]\emph{Phi Beta Kappa} (on junior standing), The College of Wooster, Wooster, Ohio
%\item[1992]Elizabeth Sidwell Wagner Prize in Mathematics, The College of Wooster
%\item[1992]William H. Wilson Prize in Mathematics, The College of Wooster
%\item[May 11, 1992]B.A., Mathematics, The College of Wooster
%\item[1997]Finalist for Graduate Teaching Award, The Ohio State University, Columbus, Ohio
%\item[June 21-25, 1998]Participant in the AMS-IMS-SIAM Summer Research Conferences: q-Series, Combinatorics, and Computer Algebra, Mt. Holyoke, Massachusetts
%\item[October 1998--October 1999]Graduate student representative to The Ohio State University Department of Mathematics Graduate Studies Committee, Columbus, Ohio
%\item[January 1999]q-series seminar address, The Ohio State University, Columbus, Ohio
%\item[2000]Finalist for Departmental Teaching Award, The Ohio State University, Columbus, Ohio
%\item[2000]Nominated for Graduate Teaching Award, The Ohio State University, Columbus, Ohio
%\item[April 2000]Invited colloquium talk at The College of Wooster, Wooster, Ohio
%\item[1992-- present]Graduate Teaching and Research Associate, The Ohio State University
%\end{datelist}
%
%%%This is for any publications you might have.%%%%%

%\begin{publist}  
%\pubitem{\quad}
%\pubitem{\quad}
%\end{publist}

\begin{fieldsstudy} 
    \majorfield{Computer Science and Mathematics}
	\minorfield{Music}
%    \specialization{Algorithmic Compositions}
    %\begin{studieslist}
   %\studyitem{Abstract Algebra}{Hampton}
   %\end{studieslist}
  \end{fieldsstudy}
\end{vita}

%%%%%%%%%%%%%%%%%%%%%%%%%%%%%%%%%%%%%%%%%%%%%%%%%%%%%%%
%
%  We now create the contents page and if necessary the list of figures and list of tables.
%
%%%%%%%%%%%%%%%%%%%%%%%%%%%%%%%%%%%%%%%%%%%%%%%%%%%%%%%


\cleardoublepage
\phantomsection
\addcontentsline{toc}{chapter}{Contents}

\tableofcontents
\listoffigures %Use if you have a list of figures.
\listoftables%Use if you have a list of tables.
\lstlistoflistings% Use if you are using the code option

%%%%%%%%%%%%%%%%%%%%%%%%%%%%%%%%%%%%%%%%%%%%%%%%%%%%%%%

% \input{chapters/preface} % most theses do not have a preface so this should be commented

%%%%%%%%%%%%%%%%%%%%%%%%%%%%%%%%%%%%%%%%%%%%%%%%%%%%%%%
\mainmatter

% Thesis chapters, each in a separate file

%!TEX root = ../username.tex
\chapter{Background} \label{bg}

This section contains the background information necessary to understand the project. % TODO make this more formal

%\section{Mathematical Notation} \label{bg:mathNotation}
% contains notation that may be unfamiliar to readers
% this secion may not be necessary



\section{Musical Terminology and Notation} \label{bg:musicTerminology}
% Contains musical terms and notation that may be unfamiliar to most readers

% tonic chord tone

\section[Markov Chains]{Markov Chains} \label{bg:markov}

\subsection{Definition} \label{bg:markov:definitions}

A Markov chain is a type of discrete-time stochastic process, which means a Markov chain is a sequence of random variables $\boldsymbol{X} = \{X_{n} | n \in I\}$ for some index set $I$.
Additionally, Markov chains have the special property that they depend only on the immediate past state(s).
That is, for a first-order Markov chain at time $n$, $$P(X_{n} = j \mid X_{0} = i_{0}, \ldots, X_{n - 1} = i_{i - 1}) = P(X_{n} = j \mid X_{n - 1} = i_{n - 1})$$ for a particular possible outcome $j$ of $X_{n}$ \cite{nierhaus_algorithmic_2009}.

This idea can also extend to higher-order Markov chains.
A higher-order Markov process considers more than the single most recent state to determine the next state.
An $n$th order Markov chain uses the previous $n$ states as the input to find the next state.

\subsection{Representations} \label{bg:markov:representations}

We can think of a Markov chain as a directed graph, where each state is a node, each edge is a transition between states, and the probabilities of transitioning between states are represented by the edge weights.
See Figure \ref{fig:markovGraph} for a visual representation of this idea.

\begin{figure}[h]
	\centering
	\includegraphics[]{figures/markovGraph.png} % TODO: Make a better diagram
	\caption{A Markov chain represented as a graph.}
	\label{fig:markovGraph}
\end{figure}

When implementing a Markov chain, however, it is perhaps easier to represent it as an $(n + 1)$-dimensional array, where $n$ is the order of the Markov chain.
We call this $(n + 1)$-dimensional array the transition matrix.
It contains the probabilities of transitioning from one state to another.
See Figure \ref{fig:markovMatrix} for an example of the same Markov chain as in Figure \ref{fig:markovGraph} in matrix form.

\begin{figure}[h]
	\centering
	\begin{tabular}{c | c c c}
		& $A_{1}$ & $B_{1}$ & $C_{1}$\\
		\hline
		$A_{0}$ & $0.3$ & $0.5$ & $0.2$\\
		$B_{0}$ & $0.6$ & $0.2$ & $0.2$\\
		$C_{0}$ & $0.4$ & $0.1$ & $0.5$
	\end{tabular}
	\caption{A Markov chain represented as a transition matrix.}
	\label{fig:markovMatrix}
\end{figure}

\subsection{Limitations} \label{bg:markov:limitations}

A major limitation of Markov chains is their inability to generate truly novel output.
In order for some state to appear, a transition to that state from the previous state must appear in the source material.
That is, no truly novel transitions may appear; all transitions that appear in the output of the Markov chain must have appeared somewhere before.
Additionally, lower-order chains may produce nonsensical output, whereas a chain of sufficiently high order will exactly copy the source material.
Another limitation is that the process may get stuck in a ``local loop''.
This may happen when the chain proceeds to a state which only transitions to itself, or transitions to a set of states that only transition to each other.

%!TEX root = ../username.tex
\chapter{A Markov Model for Melody Generation} \label{markov}

\section{Setup of the Model} \label{markov:setup}

In this model, we use two Markov processes: one to determine interval changes between notes to generate pitches, and the other to generate rhythms.
To create the interval transition matrix, we iterate over the source -- in our case a corpus of music by J.S. Bach -- keeping track of the previous interval at each step and incrementing the matrix at position $a_{i,j}$, where $i$ is the previous interval and $j$ is the current interval between notes.
Note that for a matrix $m$, $m_{i,j}$ indicates the element in row $i$ and column $j$.
The rhythmic transition matrix is similarly generated.
While iterating over the source to generate the interval transition matrix, we can also build the rhythmic transition matrix.
To do this we keep track of the duration of the previous rhythm at each step and increment the transition matrix at position $b_{k,l}$ where $k$ is the previous rhythmic duration and $l$ is the current rhythmic duration.

Because the number of possible intervals and rhythms are theoretically infinite and the transition matrix would be very sparse, we do not want to try to store a matrix capable of holding every possible transition.
Instead, we can use a hash table to only store transitions that actually appear in the source.

An example rhythmic transition matrix is given in Figure \ref{fig:rhythmTransitionMatrix}, based on the \textit{Little Fugue in G Minor} by J.S. Bach.

\begin{figure}
	\centering
	\begin{tabular}{c | c c c c c c c c}
		& 0.25 & 0.5 & 0.75 & 1.0 & 1.25 & 1.5 & 2.0 & 2.25\\
		\hline
		0.25 & 606 & 21 & 1 & 2 & 1 & 2 & 0 & 1\\
		0.5 & 23 & 109 & 0 & 7 & 1 & 0 & 0 & 0\\
		0.75 & 1 & 0 & 0 & 0 & 0 & 0 & 0 & 0\\
		1.0 & 1 & 6 & 0 & 3 & 0 & 3 & 2 & 0\\
		1.25 & 2 & 0 & 0 & 0 & 0 & 0 & 0 & 0\\
		1.5 & 0 & 5 & 0 & 0 & 0 & 0 & 0 & 0\\
		2.0 & 0 & 0 & 0 & 2 & 0 & 0 & 0 & 0\\
		2.25 & 1 & 0 & 0 & 0 & 0 & 0 & 0 & 0
	\end{tabular}
	\caption{An example rhythmic transition matrix. A value of $1.0$ indicates a quarter note.}
	\label{fig:rhythmTransitionMatrix}
\end{figure}

\section{Generation} \label{markov:generation}

After the transition matrices are created, the program randomly selects a rhythmic value, based on the probabilities that correspond with the previous note, and a pitch.
The pitch value comes from the interval from the previous note.
The interval is chosen based on the probabilities that correspond with the previous interval.

The program stops generating notes when the stopping criteria are met: the generated melody is at least eight measures, the melody ends at the end of a measure, and the last note is a tonic chord tone.
Because the generated melody could theoretically be infinite, the program terminates generation after one-hundred measures are generated whether or not the stopping criteria are met.

%!TEX root = ../username.tex
\chapter{Genetic Algorithms} \label{ga}

In this project, the melodies generated by the Markov model are used as the initial population for the genetic algorithm.

\section{Fitness Function} \label{ga:fitness}

Several previous papers discuss the use of interactive fitness functions \cite{papadopoulos_ai_1999, mcvicar_autoguitartab:_2015}.
With this method, a human listens to the melodies and selects the best from each generation to be used as the parents of the next generation.
This method does well at picking the most pleasing music to human ears, but it requires humans to listen to the music, which is slow and can lead to fatigue on the part of the evaluators.

Rather than rely on humans who may become tired, and may not be completely objective, we want to use an automated fitness function.
Some authors discuss techniques for fitness functions as applied to music \cite{papadopoulos_ai_1999, de_freitas_originality_2011, alfonseca_fitness_2006}.
For example, Alan de Freitas and Frederico Guimaraes use a fitness function that penalizes any note outside of the C Major scale \cite{de_freitas_originality_2011}, while George Papadopoulos and Geraint Wiggins use a fitness function that considers several characteristics of the melody, including consecutive intervals, note durations, and melodic contour \cite{papadopoulos_ai_1999}.
Music from the common practice period of Western music, which lasted from the late Baroque period through the Romantic period (1650-1900), generally followed a complex set of rules regarding harmony, rhythm, and duration.
We could manually define a function that applies some penalty for breaking one of these rules, but this approach could be highly error prone and require lots of manual tweaking.
Additionally, if we want to create music resembling a different era, we would have to write a new fitness function.
% Apparently, George Papadopoulos shares the same name as someone who plead guilty to lying to the FBI. What an unfortunate name to share.

Rather, a fitness function can be approximated using what we call a \textit{surrogate model}.
A surrogate model is used in contexts where direct measurement is computationally expensive or otherwise difficult to define.
In this project we use an artificial neural network as the surrogate model.
We expect the neural network to pick up on which rules are the most important, altering its weights accordingly.
Because music has temporal properties, the type of neural network used in this project is a \textit{long short-term memory} (LSTM) neural network.
An LSTM network is a special type of \textit{recurrent neural network}, where nodes link back to themselves.
This property of saving information about previous elements in the input sequence allows the LSTM neural network to build relationships between parts of the inputs that are far apart in a time sequence.
Figure \ref{fig:lstm_cell} shows the idea that the input to the LSTM neuron is kept in subsequent iterations by the neuron passing a value back to itself.

\begin{figure}[h]
	\centering
	\includegraphics{figures/lstm_cell.pdf}
	\caption{An LSTM Neuron}
	\label{fig:lstm_cell}
\end{figure}

We have two classes of training data: ``good'' melodies, which were written by a real composer, and ``bad'' melodies, which are entirely randomly generated before training the fitness function.
We expect the network might pick up on attributes such as how long a note lasts, what the pitch of a note is, and its relation to the notes around it.
Notes can be related to those around them in terms of pitch -- pitches might form a broken chord -- or in terms of rhythm -- eighth notes or sixteenth notes often appear in groups.
However, we cannot know for certain what the network decides is important before training.

In our corpus of music, the sixteenth note is most commonly the smallest rhythmic value in a piece of music, so we first preprocess the music data by converting all rhythms to consecutive sixteenth notes.
The motivation for this method, rather than trying to encode the rhythm in another way comes from Peter Todd \cite{todd_connectionist_1989}.
See Section \ref{software:ga:fitness} for more information about how the data is presented to the neural network and the configuration of the network.

During training, in addition to samples of correct music from our corpus, we also feed the network samples of random noise as incorrect examples.

Using this network setup, we regularly achieved $99\%$ accuracy determining which samples are real music and which are random notes.

A detailed description of the software implementation of the fitness function is in Section \ref{software:ga:fitness}

\section{Mutations} \label{ga:mutate}

There are several natural mutations to consider when working with music.
Two common compositional techniques are inversion and retrograde.
Inversion takes a section of music, generally a couple of measures, and inverts all the intervals, so an interval up becomes the same interval down.
Retrograde reverses the order of one or both of the rhythms and pitches of a section of music.
Retrograde and inverse can also be combined, to yield retrograde-inverse.
Typically, the section is inverted, and the new notes are then read backwards, though some composers, such as Igor Stravinsky, reversed the order of the notes first.
See Figure \ref{fig:p-r-i-ri} for an example of inversion, retrograde, and retrograde-inversion.
These techniques are especially common in twelve tone music, which was developed during the early twentieth century by Arnold Schoenberg, though they can also be found in other types of music.
An example of a twelve tone piece is Schoenberg's \textit{Wind Quintet} Op. 26.
In our genetic algorithm, we apply these compositional techniques to the melodies from the previous generation to produce new candidates for the current generation.

\begin{figure}
	\centering
	\includegraphics[width=\linewidth]{figures/P-R-I-RI.png} % TODO cite image https://commons.wikimedia.org/wiki/File:P-R-I-RI.png
	\caption{Common  clockwise from the top left: prime (original form), retrograde, inverse, and retrograde-inverse.}
	\label{fig:p-r-i-ri}
\end{figure}

An important idea when working with genetic algorithms is \textit{crossover}, where members of the population are spliced together to create new candidates.
In crossover, the fittest members of the previous generation are ``mated'' by choosing points and 
Finally, to maintain diversity in the population, we introduce some random changes to the population at a rate determined by some volatility factor, which can either stay constant or change over time.
% TODO expand on crossover

%\section{Manipulating Melodies} \label{ga:manip}


A detailed description of the genetic algorithm software is in Section \ref{software:ga}.
%!TEX root = ../username.tex
\chapter{Mathematics In Music} \label{}

% discuss interval relations (P8, P4, etc), why there is no definitive Hz value for any pitch -- it's all relative
% rational multiples of each other
% tuning systems

% overtones

% prove pitch classes in equally tempered octave form an abelian group with 12 elements + other set and group theory related topics

% spectrogram

%!TEX root = ../username.tex
\chapter{The Software} \label{software}

% TODO give more details

This chapter discusses the various components of the software used in this project.

All components use \texttt{music21}, a Python library for music developed by MIT, to handle input and output of music notation to and from a Python-native format.
Input files are in MIDI format.

\section{Overview} \label{software:overview}

The data flow for the set of programs used in this project are as follows:
The Markov chain program reads a corpus of music to create its transition matrices.
These Markov chains are then used to produce the initial population for the genetic algorithm.
The genetic algorithm uses a long short term memory artificial neural network as a surrogate fitness function, which was trained with its own music corpus.
The final output from the genetic algorithm then is sent to the harmonization program, which creates other musical voices to accompany the generated melody.

% TODO add diagram of data flow

\section{Markov Melody Generator} \label{software:markov}

In the code repository for this project, the directory \texttt{intervalMarkovChain} contains Python code to generate melodies using Markov chains of arbitrary order.
The theory behind this software component is discussed in Section \ref{bg:markov} and Chapter \ref{markov}.
The software uses two Markov chains to generate a melody, one for the intervals between notes, and the other for rhythms of notes.

The file \texttt{intervalMarkovChain/markovChain.py} contains a class to create these Markov chains.
When a \texttt{MarkovChain} object is created, it requires an order to use.
By default, the order is $1$.
To create the transition matrix, the function \texttt{create\_transition\_matrix(streams, chainType)} accepts a list of the \texttt{music21} \texttt{Streams} to use as the training source, as well as what type of Markov chain it should be -- should it calculate transition probabilities for rhythms or for the intervals between notes?

An $n$th order Markov chain is represented using \texttt{defaultdict}s nested $n + 1$ deep.
Two interesting functions in this class are \texttt{arbitrary\_depth\_dict\_get} and \texttt{arbitrary\_depth\_dict\_set}.
These functions allow one to read from and write to \texttt{dict}s of arbitrary depth.
These are necessary to be able to create and use the transition matrices because the order of the Markov chain may not always be the same, so the number of index operators desired cannot be hard-coded.

The function \texttt{arbitrary\_depth\_dict\_get(subscripts, default, nested\_dict)} works by recursively accessing \texttt{nested\_dict} with the first element of \texttt{subscripts} until either \texttt{subscripts} is empty or a key from \texttt{subscripts} does not exist.
In the former case, the function returns the value of \texttt{nested\_dict} specified by \texttt{subscripts}.
In the latter case, the function returns the specified default value.

The function \texttt{arbitrary\_depth\_dict\_set(subscripts, \_dict, val)} works by iterating through the \texttt{subscripts} of \texttt{\_dict}, going a level deeper at each step.
At each level, an empty dictionary is created if one did not already exist at that index.
This function makes use of the fact that the \texttt{dict} data structure in Python can be assigned to a new variable by reference in order to go into the next index.
On the final index, the dictionary is assigned the value at the (possibly nested) index as specified by \texttt{subscripts}.

The file \texttt{intervalMarkovChain/intervalMarkovChain.py} contains code to read a corpus of music, generate the Markov chains using the \texttt{MarkovChain} class, and create a new melody with them.
Because a Markov chain requires some predetermined seed notes to refer to, the first four notes of each generated melody are C, D, E, and D.
These pitches were chosen because the intervals between them (M2 and M-2) will appear in almost any corpus of music at least once.
To keep some rhythmic diversity, the duration for each of these four notes is randomly chosen to be a sixteenth note, eighth note, or quarter note.
The decision to include four seed notes was subjectively made by the author's thoughts that melodies produced using a fourth order interval Markov chain were note sufficiently better than melodies produced using a third order chain to warrant the extra computational cost.
In order to use a Markov chain with an order higher than four, more seed notes must be provided.

\section{Genetic Algorithms} \label{software:ga}

The directory \texttt{geneticAlgorithms} contains Python code to manipulate existing melodies with genetic algorithms to produce better output.
The theory behind this software component is in Section \ref{bg:ga} and Chapter \ref{ga}.

\subsection{Fitness Function} \label{software:ga:fitness}

This program uses an LSTM neural network created using TensorFlow as a surrogate fitness function.
TensorFlow is a machine learning library for Python developed by the Google Brain Team.
The file \texttt{geneticAlgorithms/lstm.py} contains code to convert music data to a form usable by the artificial neural network, train the LSTM neural network on that modified music data, and save the generated model for later use.

The first step is to read a corpus of music to train the function on.
If no music corpus is provided, the program uses the collection of Bach chorales that come with music21.
As with the Markov chain module, this module normalizes the melodies to the key of C before training.
This removed the need to take the key a piece is into account.

After the training input is normalized it is converted to a form the artificial neural network will be able to recognize.
This implementation splits the music into a series of sixteenth notes with pitch the pitch set to a MIDI value from $0$ to $127$.
Now that the good music data is in a form the neural network can recognize, a number of random sequences of sixteenth notes is generated to act as samples of bad samples.
We want the neural network to train on an evenly distributed sampling of good and bad data, so we shuffle the list of data using Python's \texttt{random.shuffle()} method.

Now the program is ready to create and train the LSTM network.
As discussed in Section \ref{ga:fitness}, we create an LSTM network that has $128$ input neurons, $256$ neurons the input is run through, and two output neurons.
To train the model, we run the training data through a function to reshape it to a $N$ x 64 x 128 list, where $N$ is the number of training samples.
$70\%$ of this reshaped data is then fed through the neural network for five epochs.
That is, the data is fed to the network five times.
At this point, the accuracy of the model is found by evaluating the remaining $30\%$ of the training data and comparing the output to the actual labels.
If the accuracy is sufficient, the TensorFlow session and the model are returned to be used on new data later.

\subsection{Mutations} \label{software:ga:mutations}

The file \texttt{geneticAlgorithms/genetic.py} contains functions to perform some operations on music21 \texttt{Streams}.
Section \ref{ga:mutate} contains the theoretical background of these mutations.

This file provides six functions that can be used to operate on music21 \texttt{stream} objects.
These functions are \texttt{transpose()}, \texttt{inverse()}, \texttt{retrograde()}, \texttt{retrograde\_inverse()}, \texttt{inverse\_retrograde()}, and \texttt{crossover()}.

\texttt{transpose(stream, interval)} is simply a wrapper for music21's \texttt{Stream.transpose()} method.
It accepts a \texttt{Stream} and an amount to transpose by, measured in semitones.

\texttt{inverse(stream)} returns the inverse of \texttt{stream}.
That is, it takes the inverts the intervals between notes.
For example, the sequence CDE becomes CB$\flat$A$\flat$.

\texttt{retrograde(stream, reverse\_notes, reverse\_rhythms)} returns the retrograde version of \texttt{stream}, with the options to reverse the pitches, the rhythms, or both.

\texttt{retrograde\_inverse(stream)} and \texttt{inverse\_retrograde(stream)} are wrapper functions for \texttt{inverse()} and \texttt{retrograde()}.

\texttt{crossover(parent1, parent2, crossover\_points)} provides the ability to perform crossover of two \texttt{Stream}s.
The function returns two new \texttt{Stream}s.
The first \texttt{Stream} we return is obtained by copying notes from \texttt{parent1} until the first index in \texttt{crossover\_points} is reached, then copying notes from \texttt{parent2} until the second index in \texttt{crossover\_points}, and so on, alternating which parent's notes are copied, until there are no more indices in \texttt{crossover\_points}.
The second returned \texttt{Stream} is obtained similarly, but starts by copying notes from \texttt{parent2}.

\subsection{Evolving Melodies} \label{software:ga:evolving}

The file \texttt{geneticAlgorithms/geneticAlgorithms.py} contains code to carry out the genetic algorithm that takes some initial population of melodies, remixes and mutates them, and eventually produces some more fit output as measured by the fitness function.
It includes uses all of the previously discussed components to achieve this.

The program starts by computing the fitness function using \texttt{lstm.py}.
Then the initial population is produced.
This can either use \texttt{intervalMarkovChain/intervalMarkovChain.py} to generate the initial population, or use completely random music to accomplish this.
Starting with the Markov chain will generally produce an initial population with a higher fitness, but it will take more time to generate the population.
The fitness values of the initial population are calculated while the initial population is being generated.

At this point the program is ready to enter the main loop.
Its stopping conditions are when a melody has a high enough fitness, or the maximum number of generations has been reached.
Inside the loop, the program uses the \texttt{multiprocess} library to create a pool of workers to remix the population using the functions inside of \texttt{mutations.py}.
% TODO explain why the choice to use multiprocess was made
After this remixing, the loop generates some melodies using crossover, and mutates some of the melodies.
This mutation is to ensure the population does not stagnate.
At this point the loop calculates the fitnesses of the new population, chooses the seed population for the next generation, and goes back to the beginning of the loop.

% \section{Harmonization and Counterpoint}


\section{How to Use the Software} \label{software:howtouse}

\subsection{Markov Melody} \label{software:howtouse:markov}

To produce a melody using the Markov Melody program, call the program \texttt{intervalMarkovChain.py} along with the corpus of music to use, either directly as the MIDI files to use, or as a directory containing MIDI files.
If no corpus is provided, it defaults to any \texttt{.mid} files located in the relative directory \texttt{../corpus}.
From the command line, this might look like

\texttt{python3 intervalMarkovChain.py song1.mid song2.mid}

\noindent when directly specifying the MIDI files to use, or

\texttt{python3 intervalMarkovChain.py}

\noindent to use the default music corpus.
The function \texttt{generate\_melody()} also checks if the parameter \texttt{corpus} is set to \texttt{False}.
In this case that it is \texttt{False}, the function uses the built in music21 collection of Bach chorales.
This option is only available when importing the file from another Python script.


\subsection{Genetic Melody} \label{software:howtouse:ga}

Run \texttt{python3 geneticAlgormithm.py} to generate a set of melodies produced by the genetic algorithm.
The program accepts some optional flags:

\noindent -f, -{}-desired-fitness value 

Sets the fitness required to terminate the program to value

\noindent -s, -{}-population-size value

Sets the size of the population of each generation to value

\noindent -g, -{}-max-generations value

Set the maximum number of generation the program will run to value

\noindent -m

Sets the program to use Markov chains to generate the initial population

For example: To use $5000$ generations with a population of $500$ and a desired fitness of $25$, with the initial population produced using Markov chains, you could use

\texttt{python3 geneticAlgorithm.py -f 25 -g 5000 -s 500 -m}

\noindent or

\texttt{python3 geneticAlgorithm.py --desired-fitness 25 --max-generation 5000 --population-size 500 -m}


%%%%%%%%%%%%%%%%%%%%%%%%%%%%%%%%%%%%%%%%%%%%%%%%%%%%%%%
%
%  This section starts the back matter. The back matter includes appendices, indicies, and the
%  bibliography
%
%%%%%%%%%%%%%%%%%%%%%%%%%%%%%%%%%%%%%%%%%%%%%%%%%%%%%%%

\backmatter

% \input{appendices/math}

%%%%%%%%%%%%%%%%%%%%%%%%%%%%%%%%%%%%%%%%%%%%%%%%%%%%%%%
%
%  This section would be used if you are not using BibTeX. Look at Kopka and Daly for how to
%  format a bibliography manually as well as how to use BibTeX.
%
%%%%%%%%%%%%%%%%%%%%%%%%%%%%%%%%%%%%%%%%%%%%%%%%%%%%%%%

%\begin{thebibliography}{99}
%\bibitem{}
%\bibitem{}
%\end{thebibliography}

%%%%%%%%%%%%%%%%%%%%%%%%%%%%%%%%%%%%%%%%%%%%%%%%%%%%%%%
%
%  We used BibTeX to generate a Bibliography. I would recommend this method. However, it is
%  not required.
%
%%%%%%%%%%%%%%%%%%%%%%%%%%%%%%%%%%%%%%%%%%%%%%%%%%%%%%%

\renewcommand\bibname{References} % changes the name of the Bibliography

\nocite{*} % This command forces all the bibliography references to be printed -- not just 
              % those that were explicitly cited in the text.  If you comment this out, the bibliography
              % will only include cited references.
\ifthenelse{\boolean{woosterchicago}}{
\bibliographystyle{woosterchicago}}{\ifthenelse{\boolean{achemso}}{
\bibliographystyle{achemso}}{\bibliographystyle{plainnat}}}
% if you have used the woosterchicago class option then your references and citations will be in Chicago format. If you have used the achemso class option then your references and citations will be in the American Chemical Society format. If you do not specify a citation format then the default Wooster format will be used.
\bibliography{references} % load our Bibliography file

%%%%%%%%%%%%%%%%%%%%%%%%%%%%%%%%%%%%%%%%%%%%%%%%%%%%%%%
%
%                                                                Index
%
%  Uncomment the lines below to include an index. To get an index you must put 
%  \index{index text} after any words that you want to appear in the index.
%  Subentries are entered as \index{index text!subentry text}. You must also run the
%  makeindex program to generate the index files that LaTeX uses. The PCs are set to run
%  makeindex automatically.
%
%%%%%%%%%%%%%%%%%%%%%%%%%%%%%%%%%%%%%%%%%%%%%%%%%%%%%%%

%\ifthenelse{\boolean{index}}{
%\cleardoublepage
%\phantomsection
%\addcontentsline{toc}{chapter}{Index}
%\printindex}{}

%%%%%%%%%%%%%%%%%%%%%%%%%%%%%%%%%%%%%%%%%%%%%%%%%%%%%%%
%
%                                                                Colophon
%
%  A Colophon is a section of a printed document that acknowledges the designers and printers of the work.
% The colophon also includes information about the fonts and paper used in the printing. It is not required 
% for your IS and can be commented out.
%
%%%%%%%%%%%%%%%%%%%%%%%%%%%%%%%%%%%%%%%%%%%%%%%%%%%%%%%

%\ifthenelse{\boolean{colophon}}{
%\begin{colophon}
%This Independent Study was designed by Dr. Jon Breitenbucher.\newline
%It was edited and set into type in Wooster, Ohio,\newline
%using the \ifthenelse{\boolean{xetex}}{\XeTeX\ typesetting system designed by Jonathan Kew}{\LaTeX\ typesetting system designed by Leslie Lamport}\newline
%and based on the original \TeX\ system of Donald Knuth.\newline
%It was printed and bound by Office Services at The College of Wooster.
%
%The text face is Adobe Garamond Pro, designed by Robert Slimbach.\newline
%This is the Opentype version distributed by Adobe Systems\newline
%and purchased as part of the Adobe Type Classics for Learning.
%
%The paper is standard laser copier paper and not of archival quality.
%\end{colophon}}{}
\clearpage\thispagestyle{empty}\null\clearpage
\end{document}