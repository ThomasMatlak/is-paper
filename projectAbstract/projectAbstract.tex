\documentclass[12pt]{article}

\title{Algorithmic Composition: Project Abstract}
\author{Thomas Matlak}
\date{14 September 2017}

\begin{document}
	\maketitle
	
	The goal of this project is to investigate various techniques used in the algorithmic composition of music. The project will cover the theory behind the algorithms used, as well as the mathematical theories behind what makes music that sounds ``good.'' Results of the project will be software to generate musical melodies and counterpoint and a paper detailing the implementation and theory behind the programs. The project will result in learning about various facets of algorithmic composition, from the algorithm design to measuring the quality of the music produced.
	
	The programming portion of the project will produce programs to generate melody and counterpoint. Programming for the project will primarily be done in Python. I anticipate my personal computers being sufficient for any necessary computational power.
	
	Challenges in the project include finding or creating a good representation of music within Python, and ensuring input and output files conform to notational standards. Potential problems in the project include the chosen representation of music being insufficient for easily using music data as input to an algorithm, and an algorithm being investigated proving ill-fit to generate melody or counterpoint. The first potential problem can be overcome by altering the music representation as needed. Depending on how late the second problem occurs, it might be solved by pivoting the project to use a different algorithm.
	
	The parts of the project due in the first semester are the abstract and annotated bibliography in mid-September, a presentation focused on the computer science portion in early October, the thesis outline in mid-November, and a presentation focused on the math portion in early December. By the end of the first semester, the portion of the project on generating melody should be complete.
	
	The parts of the project due in the second semester are the first complete draft in early March, the final draft before Spring Break, the final thesis the Monday after Spring Break, digital copies of the project a poster the week after Spring Break, the poster presentation at I.S. Symposium in April, and an oral defense after Spring Break. Research in the second semester will focus primarily on generating counterpoint.
	
	% 5 sources
	\nocite{*}
	
	\bibliographystyle{plain-annote}
	\bibliography{references}
\end{document}
